% Options for packages loaded elsewhere
\PassOptionsToPackage{unicode}{hyperref}
\PassOptionsToPackage{hyphens}{url}
\PassOptionsToPackage{dvipsnames,svgnames,x11names}{xcolor}
%
\documentclass[
  letterpaper,
  DIV=11,
  numbers=noendperiod]{scrartcl}

\usepackage{amsmath,amssymb}
\usepackage{iftex}
\ifPDFTeX
  \usepackage[T1]{fontenc}
  \usepackage[utf8]{inputenc}
  \usepackage{textcomp} % provide euro and other symbols
\else % if luatex or xetex
  \usepackage{unicode-math}
  \defaultfontfeatures{Scale=MatchLowercase}
  \defaultfontfeatures[\rmfamily]{Ligatures=TeX,Scale=1}
\fi
\usepackage{lmodern}
\ifPDFTeX\else  
    % xetex/luatex font selection
\fi
% Use upquote if available, for straight quotes in verbatim environments
\IfFileExists{upquote.sty}{\usepackage{upquote}}{}
\IfFileExists{microtype.sty}{% use microtype if available
  \usepackage[]{microtype}
  \UseMicrotypeSet[protrusion]{basicmath} % disable protrusion for tt fonts
}{}
\makeatletter
\@ifundefined{KOMAClassName}{% if non-KOMA class
  \IfFileExists{parskip.sty}{%
    \usepackage{parskip}
  }{% else
    \setlength{\parindent}{0pt}
    \setlength{\parskip}{6pt plus 2pt minus 1pt}}
}{% if KOMA class
  \KOMAoptions{parskip=half}}
\makeatother
\usepackage{xcolor}
\setlength{\emergencystretch}{3em} % prevent overfull lines
\setcounter{secnumdepth}{-\maxdimen} % remove section numbering
% Make \paragraph and \subparagraph free-standing
\ifx\paragraph\undefined\else
  \let\oldparagraph\paragraph
  \renewcommand{\paragraph}[1]{\oldparagraph{#1}\mbox{}}
\fi
\ifx\subparagraph\undefined\else
  \let\oldsubparagraph\subparagraph
  \renewcommand{\subparagraph}[1]{\oldsubparagraph{#1}\mbox{}}
\fi

\usepackage{color}
\usepackage{fancyvrb}
\newcommand{\VerbBar}{|}
\newcommand{\VERB}{\Verb[commandchars=\\\{\}]}
\DefineVerbatimEnvironment{Highlighting}{Verbatim}{commandchars=\\\{\}}
% Add ',fontsize=\small' for more characters per line
\usepackage{framed}
\definecolor{shadecolor}{RGB}{241,243,245}
\newenvironment{Shaded}{\begin{snugshade}}{\end{snugshade}}
\newcommand{\AlertTok}[1]{\textcolor[rgb]{0.68,0.00,0.00}{#1}}
\newcommand{\AnnotationTok}[1]{\textcolor[rgb]{0.37,0.37,0.37}{#1}}
\newcommand{\AttributeTok}[1]{\textcolor[rgb]{0.40,0.45,0.13}{#1}}
\newcommand{\BaseNTok}[1]{\textcolor[rgb]{0.68,0.00,0.00}{#1}}
\newcommand{\BuiltInTok}[1]{\textcolor[rgb]{0.00,0.23,0.31}{#1}}
\newcommand{\CharTok}[1]{\textcolor[rgb]{0.13,0.47,0.30}{#1}}
\newcommand{\CommentTok}[1]{\textcolor[rgb]{0.37,0.37,0.37}{#1}}
\newcommand{\CommentVarTok}[1]{\textcolor[rgb]{0.37,0.37,0.37}{\textit{#1}}}
\newcommand{\ConstantTok}[1]{\textcolor[rgb]{0.56,0.35,0.01}{#1}}
\newcommand{\ControlFlowTok}[1]{\textcolor[rgb]{0.00,0.23,0.31}{#1}}
\newcommand{\DataTypeTok}[1]{\textcolor[rgb]{0.68,0.00,0.00}{#1}}
\newcommand{\DecValTok}[1]{\textcolor[rgb]{0.68,0.00,0.00}{#1}}
\newcommand{\DocumentationTok}[1]{\textcolor[rgb]{0.37,0.37,0.37}{\textit{#1}}}
\newcommand{\ErrorTok}[1]{\textcolor[rgb]{0.68,0.00,0.00}{#1}}
\newcommand{\ExtensionTok}[1]{\textcolor[rgb]{0.00,0.23,0.31}{#1}}
\newcommand{\FloatTok}[1]{\textcolor[rgb]{0.68,0.00,0.00}{#1}}
\newcommand{\FunctionTok}[1]{\textcolor[rgb]{0.28,0.35,0.67}{#1}}
\newcommand{\ImportTok}[1]{\textcolor[rgb]{0.00,0.46,0.62}{#1}}
\newcommand{\InformationTok}[1]{\textcolor[rgb]{0.37,0.37,0.37}{#1}}
\newcommand{\KeywordTok}[1]{\textcolor[rgb]{0.00,0.23,0.31}{#1}}
\newcommand{\NormalTok}[1]{\textcolor[rgb]{0.00,0.23,0.31}{#1}}
\newcommand{\OperatorTok}[1]{\textcolor[rgb]{0.37,0.37,0.37}{#1}}
\newcommand{\OtherTok}[1]{\textcolor[rgb]{0.00,0.23,0.31}{#1}}
\newcommand{\PreprocessorTok}[1]{\textcolor[rgb]{0.68,0.00,0.00}{#1}}
\newcommand{\RegionMarkerTok}[1]{\textcolor[rgb]{0.00,0.23,0.31}{#1}}
\newcommand{\SpecialCharTok}[1]{\textcolor[rgb]{0.37,0.37,0.37}{#1}}
\newcommand{\SpecialStringTok}[1]{\textcolor[rgb]{0.13,0.47,0.30}{#1}}
\newcommand{\StringTok}[1]{\textcolor[rgb]{0.13,0.47,0.30}{#1}}
\newcommand{\VariableTok}[1]{\textcolor[rgb]{0.07,0.07,0.07}{#1}}
\newcommand{\VerbatimStringTok}[1]{\textcolor[rgb]{0.13,0.47,0.30}{#1}}
\newcommand{\WarningTok}[1]{\textcolor[rgb]{0.37,0.37,0.37}{\textit{#1}}}

\providecommand{\tightlist}{%
  \setlength{\itemsep}{0pt}\setlength{\parskip}{0pt}}\usepackage{longtable,booktabs,array}
\usepackage{calc} % for calculating minipage widths
% Correct order of tables after \paragraph or \subparagraph
\usepackage{etoolbox}
\makeatletter
\patchcmd\longtable{\par}{\if@noskipsec\mbox{}\fi\par}{}{}
\makeatother
% Allow footnotes in longtable head/foot
\IfFileExists{footnotehyper.sty}{\usepackage{footnotehyper}}{\usepackage{footnote}}
\makesavenoteenv{longtable}
\usepackage{graphicx}
\makeatletter
\def\maxwidth{\ifdim\Gin@nat@width>\linewidth\linewidth\else\Gin@nat@width\fi}
\def\maxheight{\ifdim\Gin@nat@height>\textheight\textheight\else\Gin@nat@height\fi}
\makeatother
% Scale images if necessary, so that they will not overflow the page
% margins by default, and it is still possible to overwrite the defaults
% using explicit options in \includegraphics[width, height, ...]{}
\setkeys{Gin}{width=\maxwidth,height=\maxheight,keepaspectratio}
% Set default figure placement to htbp
\makeatletter
\def\fps@figure{htbp}
\makeatother

\KOMAoption{captions}{tableheading}
\makeatletter
\makeatother
\makeatletter
\makeatother
\makeatletter
\@ifpackageloaded{caption}{}{\usepackage{caption}}
\AtBeginDocument{%
\ifdefined\contentsname
  \renewcommand*\contentsname{Table of contents}
\else
  \newcommand\contentsname{Table of contents}
\fi
\ifdefined\listfigurename
  \renewcommand*\listfigurename{List of Figures}
\else
  \newcommand\listfigurename{List of Figures}
\fi
\ifdefined\listtablename
  \renewcommand*\listtablename{List of Tables}
\else
  \newcommand\listtablename{List of Tables}
\fi
\ifdefined\figurename
  \renewcommand*\figurename{Figure}
\else
  \newcommand\figurename{Figure}
\fi
\ifdefined\tablename
  \renewcommand*\tablename{Table}
\else
  \newcommand\tablename{Table}
\fi
}
\@ifpackageloaded{float}{}{\usepackage{float}}
\floatstyle{ruled}
\@ifundefined{c@chapter}{\newfloat{codelisting}{h}{lop}}{\newfloat{codelisting}{h}{lop}[chapter]}
\floatname{codelisting}{Listing}
\newcommand*\listoflistings{\listof{codelisting}{List of Listings}}
\makeatother
\makeatletter
\@ifpackageloaded{caption}{}{\usepackage{caption}}
\@ifpackageloaded{subcaption}{}{\usepackage{subcaption}}
\makeatother
\makeatletter
\@ifpackageloaded{tcolorbox}{}{\usepackage[skins,breakable]{tcolorbox}}
\makeatother
\makeatletter
\@ifundefined{shadecolor}{\definecolor{shadecolor}{rgb}{.97, .97, .97}}
\makeatother
\makeatletter
\makeatother
\makeatletter
\makeatother
\ifLuaTeX
  \usepackage{selnolig}  % disable illegal ligatures
\fi
\IfFileExists{bookmark.sty}{\usepackage{bookmark}}{\usepackage{hyperref}}
\IfFileExists{xurl.sty}{\usepackage{xurl}}{} % add URL line breaks if available
\urlstyle{same} % disable monospaced font for URLs
\hypersetup{
  pdftitle={Plan de recherche : Les politiciennes insistent-elles davantage que leurs homologues masculins sur les thématiques féminines lors des débats électoraux ? La présence d'une femme politique dans un débat signifie-t-elle davantage d'évocation des thématiques féminines par rapport à un débat incluant deux hommes politiques ?},
  pdfauthor={Olivia Saffioti},
  colorlinks=true,
  linkcolor={blue},
  filecolor={Maroon},
  citecolor={Blue},
  urlcolor={Blue},
  pdfcreator={LaTeX via pandoc}}

\title{Plan de recherche : Les politiciennes insistent-elles davantage
que leurs homologues masculins sur les thématiques féminines lors des
débats électoraux ? La présence d'une femme politique dans un débat
signifie-t-elle davantage d'évocation des thématiques féminines par
rapport à un débat incluant deux hommes politiques ?}
\author{Olivia Saffioti}
\date{2024-03-11}

\begin{document}
\maketitle
\ifdefined\Shaded\renewenvironment{Shaded}{\begin{tcolorbox}[enhanced, borderline west={3pt}{0pt}{shadecolor}, boxrule=0pt, interior hidden, frame hidden, sharp corners, breakable]}{\end{tcolorbox}}\fi

«~De nombreux acteurs -- électeurs, responsables de partis, candidats,
journalistes -- transfèrent leurs attentes stéréotypées à l'égard des
hommes et des femmes aux candidats hommes et femmes.~Le résultat de ces
stéréotypes est que certains traits de personnalité et certains domaines
d'expertise politique en viennent à être considérés comme féminins et
d'autres comme masculins~» (Fox et Oxley 2003, 834). Cette citation met
l'emphase sur les stéréotypes de genre et les mécanismes liant le genre
et la politique. Les postulats de la théorie de la construction sociale
du genre sont mis en avant. Cette dernière stipule que le genre est un
comportement social, que les individus adoptent lorsqu'ils
interagissent. Il serait issu de la culture (Chauvin 2021). Les femmes
et les hommes conformeraient leur manière de communiquer aux stéréotypes
de genre afin de répondre aux attentes de la société (Grebelsky-Lichtman
et Bdolach 2017, 276). Selon cette théorie, les électeurs évalueraient
les politicien(ne)s en fonction des stéréotypes de genre. Si les
politicien(ne)s n'adaptent par leur stratégie marketing aux stéréotypes,
ils/elles risqueraient un blacklash, soit, des évaluations négatives de
la part des électeurs (Rudman et Fairchild 2004; Coyle 2009, cités dans
Grebelsky-Lichtman et Katz 2019). Ainsi, les femmes et les hommes
politiques n'useraient pas des mêmes stratégies marketing afin de
convaincre l'électorat. En effet,~les femmes et les hommes répondraient
souvent aux stéréotypes induits par le rôle attribué à chaque sexe dans
la société. Ils conforteraient ainsi la vision des électeurs (Fox 1997).
Comme mis en exergue dans notre citation, cela inclurait l'image, les
caractéristiques personnelles, les thèmes de campagne et l'utilisation
d'enjeux politiques spécifiques (Fox 1997). À titre d'exemple, les
femmes politiques seraient perçues par l'électorat comme plus efficaces
dans les domaines de l'éducation, de la santé, de l'environnement, des
arts, de la protection des consommateurs ou encore dans l'aide à
apporter aux pauvres (Alexander and Andersen 1993; Koch 2000; McDermott
1998, cités dans Fox et Oxley 2003). À l'inverse, les hommes politiques
seraient considérés comme plus compétents pour résoudre des crises
militaires ou de police, des problèmes d'ordre économique ou encore des
enjeux liés au commerce. Ils seraient également identifiés comme plus
légitimes pour s'occuper du contrôle de la criminalité ou de
l'agriculture (Alexander and Andersen 1993; Koch 2000; McDermott 1998,
cités dans Fox et Oxley 2003). En revanche, nous n'avons pas trouvé de
recherches qui visent la relation entre le genre des candidats et les
thématiques abordées lors des débats électoraux. Il semble ainsi
pertinent d'étudier si les femmes politiques insistent plus sur les
thématiques «~féminines~» que les hommes et si ces thématiques sont
davantage abordées lors des débats électoraux incluant une femme
politique que lors des débats opposants deux hommes politiques. Beaucoup
de recherches se sont intéressées au contexte états-unien. Or, nous
trouvions pertinent de cibler un autre contexte occidental : la France.
Nous avons choisi la France pour plusieurs raisons. Tout d'abord, les
retranscriptions des débats électoraux français sont faciles d'accès. De
surcroît, la France a compté plusieurs femmes politiques qui se sont
présentées aux élections présidentielles et qui ont donc participé à des
débats électoraux. Enfin, se centrer sur ce nouveau contexte permettra
un apport scientifique concernant les mécanismes liant le genre et la
communication politique. Autrement dit, cela nous permettra de vérifier
si les postulats de la théorie de la construction sociale du genre sont
applicables à d'autres contextes et se vérifient à travers les
thématiques abordées au sein des débats électoraux. C'est pourquoi, nous
avons décidé d'étudier les débats électoraux de second-tour des
élections françaises de 1995 à 2017. Nous avons choisi ces débats car
ceux qui avaient eu lieu après 2019 auraient biaisé nos résultats. En
effet, la pandémie de Covid-19 demeurait un sujet très discuté par les
politiciens durant les campagnes suivant 2019. Ce contexte temporel ne
nous aurait pas permis de vérifier si la santé était une thématique plus
discutée lorsque des femmes prenaient part à un débat électoral. En
outre, parmi ces quatre débats, deux impliquent des candidates femmes~:
Ségolène Royal et Marine Le Pen. Or, ces deux politiciennes
appartiennent à des partis politiques différents~: madame Royal est
membre d'un parti de gauche (parti socialiste) et madame Le Pen est
membre d'un parti d'extrême droite (le rassemblement national). Cette
situation nous permettra de contrôler une variable~: le parti
d'appartenance. En effet, dépendamment du parti politique, les intérêts
défendus et les positionnements politiques des candidat(e)s changent. À
travers ces débats nous vérifierons si les thématiques féminines (la
santé, l'éducation, la famille, l'environnement et la protection
sociale) étaient plus abordées lors des débats auxquels madame Royal et
madame Le Pen ont pris part. Nous pourrons également vérifier s'il
s'agissait de thématiques plus importantes pour elles que pour les
hommes (si elles insistaient davantage sur ces thématiques que les
hommes politiques). Pour ce faire, nous utiliserons les bases de données
portant sur les débats électoraux suivants : celui opposant Emmanuel
Macron à Marine Le Pen en 2017, celui opposant François Hollande à
Nicolas Sarkozy en 2012, celui opposant Ségolène Royal à Nicolas Sarkozy
en 2007 et celui opposant Jacques Chirac à Lionel Jospin en 1995
(Fréchet 2024). Puis, nous nettoierons ces bases de données. Ensuite,
nous élaborerons un dictionnaire. À partir de ce dernier, nous
analyserons nos bases de données afin de vérifier à quelles fréquences
les thématiques de la santé, de l'éducation, de la famille, de
l'environnement et de la protection sociale sont évoquées par les
candidats lors des débats. Enfin, nous présenterons nos résultats.

\hypertarget{hypothuxe8se}{%
\subsubsection{\texorpdfstring{\textbf{\emph{Hypothèse}}}{Hypothèse}}\label{hypothuxe8se}}

Nous n'avons pas trouvé d'études portant sur la relation entre le genre
et les thématiques abordées lors des débats électoraux. En revanche,
plusieurs études micro et macro ont déjà montré, que les femmes
politiques ne priorisent pas les mêmes enjeux que les hommes (Paxton et
Hughes 2021). Ainsi, en Afrique Sub-Saharienne, elles listeraient la
pauvreté et les droits des femmes comme étant prioritaires contrairement
aux hommes (Clayton et al.2015, cité dans Paxton et Hughes 2021, 220).
En Inde, les femmes politiques voient l'accès à l'eau potable et les
politiques de bien-être comme un enjeu prioritaire, ce qui n'est pas le
cas des hommes politiques (Chattopadhyay et Duflo 2004, cité dans Paxton
et Hughes 2021, 222). Aux États-Unis, les femmes politiques donnent
beaucoup d'importance aux politiques liées au bien-être, à la famille,
aux enfants, à l'éducation, à la santé, ainsi qu'à la discrimination de
genre et à l'amélioration du statut économique des femmes (Bratton et
Haynie 1999; Swers 2013; Thomas 1991; Reingold et Smith 2012, cités dans
Paxton et Hughes 2021, 226).

En outre, selon la théorie de la construction sociale du genre et du
schéma du genre, les électeurs perçoivent les politiciennes comme plus
efficaces dans les domaines politiques dits «~féminins~» (Sanbonmatsu
2002; Fox et Oxley 2003). Ainsi, les électeurs pourraient voter pour des
candidats du même sexe qu'eux car ils défendraient leurs intérêts
(Aalberg et Todal Jenssen 2007). Par exemple, les femmes voteraient pour
des politiciennes car elles partageraient des intérêts communs relatifs
à l'éducation, à l'émancipation des femmes, ou à d'autres domaines liés
à l'expertise de la gent féminine. Dépendamment du contexte, les
préoccupations politiques des citoyens peuvent être liées à des domaines
relatifs à l'expertise des femmes comme la santé ou l'environnement.
Dans cette situation, certains segments d'électeurs peuvent avoir
tendance à voter de manière stratégique en soutenant une candidate femme
(Herrnson et al.~2003). Les femmes politiques pourraient également
profiter des préoccupations de l'opinion publique pour insister sur les
thématiques «~féminines~» dans leur programme politique. Cela leur
procurerait du soutien électoral supplémentaire, les politiciennes étant
évaluées plus positivement lorsqu'elles insistent sur les thématiques
politiques «~féminines~» durant leur campagne (Kahn 1996). Ces mêmes
préoccupations des électeurs pourraient influencer les thématiques
abordées lors des débats électoraux. En débat électoral, les
politiciennes pourraient également davantage insister sur ces
thématiques que leurs homologues masculins. Cela leur permettrait de
mettre l'emphase sur le fait que leur agenda politique est différent de
ceux de leurs adversaires hommes (Aalberg et Todal Jenssen 2007, 22).
Elle se conformeraient aux stéréotypes de genre en montrant qu'elles
vouent beaucoup d'importance à ces thématiques ``féminines''. L'objectif
serait d'obtenir davantage de soutien électoral.

C'est pourquoi, nous émettons l'hypothèse suivante~: dans un débat
électoral, lorsqu'une femme politique est présente, les thématiques
«~féminines~» sont davantage abordées. Mesdames Royal et Le Pen
insistent également davantage sur les thématiques «~féminines~» que
leurs homologues masculins.

\hypertarget{donnuxe9es-mobilisuxe9es-dans-le-cadre-de-la-recherche}{%
\subsubsection{\texorpdfstring{\textbf{\emph{Données mobilisées dans le
cadre de la
recherche}}}{Données mobilisées dans le cadre de la recherche}}\label{donnuxe9es-mobilisuxe9es-dans-le-cadre-de-la-recherche}}

Dans le cadre de notre recherche, nous avons mobilisé des données
textuelles. Nous avons utilisé les retranscriptions des quatre débats
électoraux. Nous avons trouvé ces retranscriptions en libre accès et en
format pdf sur le site internet vie-publique.fr. Ce site internet a été
mis en place par la République Française. Nous avons enregistré les
retranscriptions. Puis, nous avons converti ces documents pdf en bases
de données au format csv à l'aide du logiciel R. Toutes les
retranscriptions ont été converties en bases de données à partir des
mêmes variables : une variable «~id~» pour spécifier les candidats qui
sont opposés dans les débats (par exemple~: Royal/Sarkozy désigne que le
débat oppose Ségolène Royale à Nicolas Sarkozy). Une variable nommée
«~year~» a également été créée afin de spécifier l'année des débats, et
une variable «~date~» a été conçue afin d'identifier la date exacte
(jour/mois/année) des débats. Une autre variable intitulée «~country~» a
permis de mettre en lumière le pays dans lequel ont eu lieu les débats.
Une variable nommée «~speaker~» a également été instaurée afin de
relever qui s'exprime au sein des débats. Une autre variable intitulée
«~speaker\_turn~» met en exergue le tour de parole entre les
intervenants des débats. Nous avons également créé une variable
«~party~» qui spécifie le parti politique d'appartenance des candidats.
Elle montre également qui sont les journalistes. Enfin, nous avons
importé les textes des retranscriptions des débats sous la variable
«~text~». Les bases de données comptent donc chacune 8 variables.
Monsieur Nadjim Fréchet nous a aidé à convertir les retranscriptions en
bases de données. Nous avons mis les codes R qu'il a utilisé en annexe.

Pour mener notre recherche, nous nous centrerons sur trois variables
spécifiques~: la variable «~text~», la variable «~speaker~» ainsi que la
variable «~id~». En effet, la variable id nous permettra d'identifier
les débats dans lesquels les fréquences de mots ont été relevées. La
variable «~speaker~» permettra d'identifier à quels candidats
correspondent les fréquences de mots. En outre, ce sera sur la variable
«~text~» que nous appliquerons notre dictionnaire. Nous allons donc
nettoyer nos bases de données afin de conserver uniquement les variables
qui sont pertinentes à notre recherche. Pour cela, nous avons utilisé le
«~pipe operator~» et la fonction « select ()~» du package «~dplyr~».
Cela nous a permis de créer une nouvelle base de données comprenant
uniquement les variables «~id~», «~speaker~» et «~text~». En outre, dans
le cadre de notre étude, nous visons principalement le discours des
candidats. C'est pourquoi nous avons supprimé les éléments textuels qui
représentaient les questions/commentaires des journalistes. Pour ce
faire, nous avons utilisé la fonction «~filter ()~» du package «~dplyr~»
afin de conserver uniquement les données textuelles qui rapportent aux
paroles des candidats. Nous avons également retiré les majuscules afin
que l'analyse de dictionnaire ne soit pas biaisée~: un mot présent dans
le dictionnaire pourrait ne pas être relevé s'il contient une majuscule
dans le texte. Pour ce faire nous avons mobilisé la fonction~« tolower(
)~». Nous avons également supprimé la ponctuation afin d'uniformiser le
texte et de faciliter notre analyse de dictionnaire. En effet, la
ponctuation est inutile pour notre analyse et nous voulons éviter que
cette dernière n'influence les résultats de notre recherche. Pour l'ôter
du texte, nous avons utilisé la fonction «~gsub()~». Nous avons mis nos
codes R en annexe, avec l'exemple du débat entre Emmanuel Macron et
Marine Lepen.

\hypertarget{muxe9thode-de-la-recherche}{%
\subsubsection{\texorpdfstring{\textbf{\emph{Méthode de la
recherche}}}{Méthode de la recherche}}\label{muxe9thode-de-la-recherche}}

Afin de mener notre recherche nous allons mobiliser une méthode
quantitative. Nous allons réaliser une analyse textuelle automatisée. Ce
type d'analyse permet une lecture plus attentive des textes. L'analyse
textuelle apparaît ainsi plus précise et minutieuse (Grimmer et Stewart
2013, 268). Grâce à cette analyse, nous allons pouvoir classifier les
contenus des retranscriptions sous plusieurs catégories que nous aurons
élaborées (Grimmer et Stewart 2013, 268). Nous allons plus précisément
mener une analyse de dictionnaire. Cette méthode vise à identifier la
fréquence à laquelle des mots clés apparaissent dans un texte (Grimmer
et Stewart 2013, 274). Ces mots sont classés selon des catégories et
permettent de rendre compte du contenu des textes (Grimmer et Stewart
2013, 274).

Dans le cadre de notre recherche, nous allons créer notre propre
dictionnaire. En effet, créer notre dictionnaire permet d'inclure un
éventail de mots représentants les sujets politiques que nous souhaitons
étudier à travers notre étude. En outre, créer son dictionnaire apporte
une fiabilité interne pertinente~: les mots du dictionnaire sont
cohérents et logiques car représentent bien les thématiques politiques
que nous souhaitons étudier. Créer un dictionnaire permet également une
bonne validité convergente~: les mots inclus au sein du dictionnaire
sont utilisés/relevés dans le discours des politiciens (Nicolas et
al.~2019, 9). Autrement dit, ils sont adaptés à nos données textuelles.
Afin de créer notre dictionnaire, nous nous sommes basés sur les
dictionnaires thématiques Lexicoder d'Albugh, Quinn, Julie Sevenans et
Stuart Soroka et nous avons sélectionné certains mots. Nous avons
également rajouté des mots qui nous semblaient pertinents et qui
n'étaient pas présents dans les dictionnaires thématiques sur lesquels
nous nous sommes appuyés.

Notre dictionnaire comprend cinq catégories différentes. Tout d'abord,
nous avons une catégorie portant sur la santé. Il comprend les termes~:
« santé », « médecin », « médecins », ~« hôpital », «~hôpitaux~», «
maladie », «~maladies~», « patient », « patients », « soins », «
assurance maladie », « malades », « médicaments », « aide médicale », «
visites médicales », « visite médicale », «~sida~», «~vih~»,
«~cigarette~», «~traitement~», « traitements~», «~abus de substances~»,
«~consommation de substances~», «~vaccination~», «~vaccin~»,
«~vaccins~», «~immunisation~», «~immuniser~», «~médical~», «~médicale~»,
«~médicaux~», «~médicales~», «~infirmiers~», «~infirmières~»,
«~infirmier~», «~infirmière~», «~pharmacie~», «~pharmacies~»,
«~pharmaceutique~», «~pharmaceutiques~», «~santé mentale~», «~tabac~»,
«~médecine~», «~épidémie~», «~pandémie~», «~obésité~», «~obèse~»,
«~obèses~», «~euthanasie~», «~fécondation in vitro~», «~contraception~»,
«~contraceptions~», «~éducation sexuelle~», «~maladies sexuellement
transmissibles~», «~maladie sexuellement transmissible~», «~pilule~»,
«~mst~», «~infection sexuellement transmissible~», « infections
sexuellement transmissibles~»,~«~grippe~».

Notre deuxième catégorie porte sur la thématique de la famille, et
contient les mots~: «~enfants~», « enfant », « famille », « familles »,
« famille nombreuse », « familles nombreuses », « parents », « parent »,
« familial », « familiale », « familiales », « familiaux », «~crèche~»,
«~petite-enfance~».

Notre troisième catégorie porte sur la thématique de l'environnement.
Les termes que nous avons sélectionnés sont~: «~taxe carbone~»,
«~amiante~», «~amiantes~», «~alimentation en eau~», «~eau potable~»,
«~accès à l'eau~», «~déchets dangereux~», «~déchets radioactifs~»,
«~déchets nucléaires~», «~qualité de l'air~», «~développement durable~»,
«~gaz à effet de serre~», «~pollution~», «~déforestation~» ,
«~écologie~», «~écologique~», «~voitures électriques~», «~consommation
de pétrole~», «~carbone~», «~puits de carbones~», «~réchauffement
climatique~», «~changement climatique~», «~énergies renouvelables~»,
«~nucléaire~», «~environnemental~», «~environnementale~», «~éoliennes~»,
«~éolienne~», «~solaire~», «~solaires~», «~bio carburants», «~chaud~»,
«~chauds~», «~réchauffement planétaire~», «~uranium~», «~combustible~»,
«~combustibles~».

Notre quatrième catégorie porte sur la thématique de protection sociale,
et comprend les mots~: « retraite », «~retraites~», « protection sociale
», « protections sociales », « salariés », « indemnisation », «
indemnisations », « pauvreté », « inégalités », « allocation », «
allocations », « quotient », « aide », « aides », « assurance~»,
«~assurances~», «~banque alimentaire~», «~banques alimentaires~»,
«~appauvrir~», «~appauvrissement~», «~faible revenu~», «~revenu~»,
«~revenus~», «~sécurité sociale~», «~assurance chômage~»,
~«~bien-être~», «~pauvreté~», «~aide alimentaire~», «~aides
alimentaires~», «~service social~», «~services sociaux~», «~mère
célibataire~», «~père célibataire~», «~privation sociale~», «~privation
matérielle et sociale~», «~niveau de vie~», «~déprivation sociale~»,
«~déprivation matérielle et sociale~», «~cotisations sociales~»,
«~justice sociale~», «~assistantes sociales~», «~prestations sociales~»,
«~protection sociale~», «~tva sociale~», «~pension~», «~pensions~»,
«~chômage~», «~chômeurs~», «~insécurité sociale~», .

Enfin, notre cinquième catégorie porte sur l'éducation et inclut les
mots~: « collège», «~collèges~», « enseignement », «~élève~», « élèves
», « classe », « classes », « diplôme », « diplômes », « lycée »,
«~lycées~», « université », « universités », « universitaire », «
universitaires », « formation », « formations », « apprentissage »,
«~apprentissages~», « filière », « filières », « professionnelle », «
professionnelles », « école », «~écoles~», «~maternelle~», «~école
primaire~».

Ainsi, nous appliquerons ce dictionnaire à nos données textuelles. Le
dictionnaire permettra au logiciel R de relever la fréquence des mots
que nous venons d'énoncer. Plus précisément, il permettra de relever la
fréquence des mots dans le discours de chaque candidat. Pour ce faire,
nous utiliserons le package «~clessnverse~» de R, et nous mobiliserons
la fonction «~run\_dictionnary ()~». Une fois que nous aurons mené cette
analyse avec l'ensemble de nos textes, nous combinerons les résultats
obtenus dans une seule base de données. Cette base de données contiendra
les variables «~id~» (soit, celle qui explique quels candidats sont
opposés dans le débat) et «~speaker~», ainsi que les fréquences de mots
associées à chaque candidat pour chaque débat. Puis, nous visualiserons
les données de nos résultats à l'aide de deux graphiques en barres. Un
premier servira à comparer la fréquence des mots entre les candidats. Un
second permettra de comparer le total des fréquences de mots relevées
dans les débats (la somme des fréquences de mots relevées pour les deux
candidats dans chaque débat). Le premier graphique indiquera si les
femmes politiques insistent davantage sur les thématiques dîtes «
féminines~» que les hommes. Le deuxième graphique illustrera si la
présence d'une candidate femme dans un débat permet aux thématiques
féminines d'être davantage évoquées. Pour les générer, nous utiliserons
la fonction «~ggplot()~» présente dans le package «~tidyverse~». Ainsi,
nous pourrons identifier si lorsqu'une femme politique participe à un
débat, les thématiques politiques «~féminines~» sont traitées avec plus
d'insistance. Nous saurons également si le contenu des débats varie en
fonction du genre des candidats.

\hypertarget{bibliographie}{%
\subsubsection{\texorpdfstring{\textbf{Bibliographie}}{Bibliographie}}\label{bibliographie}}

Aalberg, Toril, et Anders Todal Jenssen. 2007. «~Gender Stereotyping of
Political Candidates: An Experimental Study of Political
Communication~». \emph{Nordicom Review} 28 (1): 17‑32.
\url{https://doi.org/10.1515/nor-2017-0198}.

Albugh, Quinn, Julie Sevenans et Stuart Soroka. 2013. \emph{Lexicoder
Topic Dictionaries}, versions de juin 2013. Montréal~: Université
McGill. \url{https://www.snsoroka.com/data-lexicoder}.

Chauvin, S.bastien.2021. . La construction sociale du genre comme
construction sociale''. \emph{Implications philosophiques,} 15 décembre
2021. \url{https://sebastienchauvin.org/category/publications/gender/}.

Fox, Richard Logan. 1997. \emph{Gender Dynamics in Congressional
Elections}. Thousand Oaks:Éditions SAGE.
\url{https://books.google.ca/books?hl=en\&lr=\&id=x2w1BfJwRkAC\&oi=fnd\&pg=PR13\&dq=gender+women+elections\&ots=8RGQYfimmw\&sig=Uo31w_YYAhrRtOEKSyA6ABiEeWw\#v=onepage\&q=gender\%20women\%20elections\&f=false}

Fox, Richard L., et Zoe M. Oxley. 2003. « Gender Stereotyping in State
Executive Elections: Candidate Selection and Success ». \emph{The
Journal of Politics} 65 (3): 833-50.
\url{https://doi.org/10.1111/1468-2508.00214}.

Fréchet, Nadjim. 2024. \emph{Bases de données portant sur les débats
électoraux entre Emmanuel Macron et Marine Le Pen (2017), François
Hollande et Nicolas Sarkozy (2012), Ségolène Royal et Nicolas Sarkozy
(2007), et Jacques Chirac et Lionel Jospin (1995).}

Grebelsky-Lichtman, Tsfira, et Liron Bdolach. 2017. « Talk like a man,
walk like a woman: an advanced political communication framework for
female politicians »\emph{. The Journal of} \emph{Legislative Studies}
23 (3): 275-300\emph{.}
\url{https://doi.org/10.1080/13572334.2017.1358979}.

Grebelsky-Lichtman, Tsfira, Roy Katz. 2019. « When a man debates a
woman: Trump vs.~Clinton in the first mixed gender presidential debate
». \emph{Revue Journal of Gender Studies} 28 (6):
699-719.\url{https://www.tandfonline.com/doi/full/10.1080/09589236.2019.1566890}

Grimmer, Justin, et Brandon M. Stewart. 2013. «~Text as Data: The
Promise and Pitfalls of Automatic Content Analysis Methods for Political
Texts~». \emph{Political Analysis} 21 (3): 267‑97.
\url{https://www.cambridge.org/core/journals/political-analysis/article/text-as-data-the-promise-and-pitfalls-of-automatic-content-analysis-methods-for-political-texts/F7AAC8B2909441603FEB25C156448F20}

Herrnson, Paul S., J. Celeste Lay, et Atiya Kai Stokes. 2003. «~Women
Running ``as Women'': Candidate Gender, Campaign Issues, and
Voter-Targeting Strategies~». \emph{The Journal of Politics} 65 (1):
244‑55. \url{https://doi.org/10.1111/1468-2508.t01-1-00013}.

Kahn, Kim Fridkin. 1996. \emph{The Political Consequences of Being a
Woman: How Stereotypes Influence the Conduct and Consequences of
Political Campaigns}. New York: Columbia University Press.

Nicolas, Gandalf, Xuechunzi Bai, et Susan Fiske. 2019. «~Automated
Dictionary Creation for Analyzing Text: An Illustration from Stereotype
Content~». \emph{Running head: automated dictionary creation}.
Department of Psychology, Princeton University.
\url{https://doi.org/10.31234/osf.io/afm8k}.

Paxton, Pamela et Melanie Hughes.2021. « Chapter 9: Do Women Make a
Difference? ». Dans \emph{Women, Politics, and Power: A Global
Perspective}. Sous la direction de Pamela Paxton, Melanie M. Hughes, et
Tiffany D. Barnes 4e éd., 218-244. Los Angeles : Pine Forge Press.
\url{https://umontreal.on.worldcat.org/oclc/1245199736}

Sanbonmatsu, Kira. 2002. « Gender Stereotypes and Vote Choice ».
\emph{American Journal of political Science} 46 (1): 20-34.
\url{https://doi.org/10.2307/3088412}.

\hypertarget{annexe}{%
\subsubsection{Annexe}\label{annexe}}

Voici les codes utilisés pour créer et nettoyer les bases de données des
textes~(exemple avec le texte du débat opposant Marine Lepen et Emmanuel
Macron).

\begin{Shaded}
\begin{Highlighting}[]
\CommentTok{\# Convertir la retranscription pdf en base de données csv }


\FunctionTok{library}\NormalTok{(pdftools)}
\end{Highlighting}
\end{Shaded}

\begin{verbatim}
Using poppler version 23.04.0
\end{verbatim}

\begin{Shaded}
\begin{Highlighting}[]
\FunctionTok{library}\NormalTok{(tidyverse)}
\end{Highlighting}
\end{Shaded}

\begin{verbatim}
-- Attaching core tidyverse packages ------------------------ tidyverse 2.0.0 --
v dplyr     1.1.4     v readr     2.1.5
v forcats   1.0.0     v stringr   1.5.1
v ggplot2   3.4.4     v tibble    3.2.1
v lubridate 1.9.3     v tidyr     1.3.0
v purrr     1.0.2     
\end{verbatim}

\begin{verbatim}
-- Conflicts ------------------------------------------ tidyverse_conflicts() --
x dplyr::filter() masks stats::filter()
x dplyr::lag()    masks stats::lag()
i Use the conflicted package (<http://conflicted.r-lib.org/>) to force all conflicts to become errors
\end{verbatim}

\begin{Shaded}
\begin{Highlighting}[]
\FunctionTok{library}\NormalTok{(lubridate)}

\NormalTok{text\_1 }\OtherTok{\textless{}{-}}\FunctionTok{suppressMessages}\NormalTok{(pdftools}\SpecialCharTok{::}\FunctionTok{pdf\_text}\NormalTok{(}\StringTok{"/Users/oliviasaffioti/Desktop/fas\_1001\_Saffioti/\_tp/TP3/macron\_vs\_lepen.pdf"}\NormalTok{))}

\NormalTok{split\_vector }\OtherTok{\textless{}{-}} \FunctionTok{paste0}\NormalTok{(}\FunctionTok{c}\NormalTok{(}\StringTok{"Mme Le Pen :"}\NormalTok{, }\StringTok{"Mme Saint{-}Cricq :"}\NormalTok{, }\StringTok{"M. Jakubyszyn :"}\NormalTok{, }\StringTok{"M. Macron :"}\NormalTok{, }\StringTok{"https"}\NormalTok{), }\AttributeTok{collapse =} \StringTok{"|"}\NormalTok{)}

\NormalTok{speaker\_names }\OtherTok{\textless{}{-}} \FunctionTok{data.frame}\NormalTok{(}\AttributeTok{text =} \FunctionTok{paste0}\NormalTok{(text\_1, }\AttributeTok{collapse =} \StringTok{" "}\NormalTok{)) }\SpecialCharTok{|\textgreater{}}
  \FunctionTok{mutate}\NormalTok{(}\AttributeTok{text =} \FunctionTok{str\_squish}\NormalTok{(}\FunctionTok{str\_replace\_all}\NormalTok{(text, }\FunctionTok{c}\NormalTok{(}\StringTok{"[:punct:]C2[:punct:]AB"}  \OtherTok{=} \StringTok{""}\NormalTok{,}
                                                   \StringTok{"[:punct:]C2[:punct:]BB"}  \OtherTok{=} \StringTok{""}\NormalTok{))),}
         \AttributeTok{text =} \FunctionTok{str\_replace\_all}\NormalTok{(text, }\FunctionTok{c}\NormalTok{(}\StringTok{"Emmanuel Macron :"}       \OtherTok{=} \StringTok{"M. Macron :"}\NormalTok{,}
                                        \StringTok{"Marine Le Pen :"}         \OtherTok{=} \StringTok{"Mme Le Pen :"}\NormalTok{,}
                                        \StringTok{"Christophe Jakubyszyn :"} \OtherTok{=} \StringTok{"M. Jakubyszyn :"}\NormalTok{,}
                                        \StringTok{"Nathalie Saint{-}Cricq :"}  \OtherTok{=} \StringTok{"Mme Saint{-}Cricq :"}\NormalTok{)),}
         \AttributeTok{speaker =} \FunctionTok{str\_extract\_all}\NormalTok{(text, split\_vector)) }\SpecialCharTok{|\textgreater{}} 
  \FunctionTok{select}\NormalTok{(}\SpecialCharTok{{-}}\NormalTok{text) }\SpecialCharTok{|\textgreater{}} 
  \FunctionTok{unnest}\NormalTok{(speaker) }\SpecialCharTok{|\textgreater{}} 
  \FunctionTok{slice}\NormalTok{(}\SpecialCharTok{{-}}\DecValTok{805}\NormalTok{)}

\NormalTok{text\_data }\OtherTok{\textless{}{-}} \FunctionTok{data.frame}\NormalTok{(}\AttributeTok{text =} \FunctionTok{paste0}\NormalTok{(text\_1, }\AttributeTok{collapse =} \StringTok{" "}\NormalTok{)) }\SpecialCharTok{|\textgreater{}} 
    \FunctionTok{mutate}\NormalTok{(}\AttributeTok{text =} \FunctionTok{str\_squish}\NormalTok{(}\FunctionTok{str\_replace\_all}\NormalTok{(text, }\FunctionTok{c}\NormalTok{(}\StringTok{"[:punct:]C2[:punct:]AB"}  \OtherTok{=} \StringTok{""}\NormalTok{,}
                                                     \StringTok{"[:punct:]C2[:punct:]BB"}  \OtherTok{=} \StringTok{""}\NormalTok{))),}
           \AttributeTok{text =} \FunctionTok{str\_replace\_all}\NormalTok{(text, }\FunctionTok{c}\NormalTok{(}\StringTok{"Emmanuel Macron :"}       \OtherTok{=} \StringTok{"M. Macron :"}\NormalTok{,}
                                          \StringTok{"Marine Le Pen :"}         \OtherTok{=} \StringTok{"Mme Le Pen :"}\NormalTok{,}
                                          \StringTok{"Christophe Jakubyszyn :"} \OtherTok{=} \StringTok{"M. Jakubyszyn :"}\NormalTok{,}
                                          \StringTok{"Nathalie Saint{-}Cricq :"}  \OtherTok{=} \StringTok{"Mme Saint{-}Cricq :"}\NormalTok{)),}
           \AttributeTok{text =} \FunctionTok{str\_split}\NormalTok{(text, split\_vector)) }\SpecialCharTok{|\textgreater{}}
  \FunctionTok{unnest}\NormalTok{(text) }\SpecialCharTok{|\textgreater{}} 
  \FunctionTok{slice}\NormalTok{(}\SpecialCharTok{{-}}\FunctionTok{c}\NormalTok{(}\DecValTok{1}\NormalTok{,}\DecValTok{806}\NormalTok{))}

\NormalTok{Data\_debate\_2017 }\OtherTok{\textless{}{-}} \FunctionTok{bind\_cols}\NormalTok{(speaker\_names, text\_data) }\SpecialCharTok{|\textgreater{}} 
  \FunctionTok{mutate}\NormalTok{(}\AttributeTok{year         =} \DecValTok{2017}\NormalTok{, }
         \AttributeTok{date         =} \FunctionTok{ymd}\NormalTok{(}\StringTok{"2017{-}05{-}03"}\NormalTok{), }
         \AttributeTok{country      =} \StringTok{"France"}\NormalTok{, }
         \AttributeTok{id           =} \StringTok{"Lepen/Macron"}\NormalTok{,}
         \AttributeTok{speaker\_turn =} \DecValTok{1}\SpecialCharTok{:}\DecValTok{804}\NormalTok{,}
         \AttributeTok{speaker =} \FunctionTok{str\_squish}\NormalTok{(}\FunctionTok{str\_replace\_all}\NormalTok{(speaker, }\FunctionTok{c}\NormalTok{(}\StringTok{"Mme"} \OtherTok{=} \StringTok{""}\NormalTok{,}
                                                         \StringTok{"\^{}M."} \OtherTok{=} \StringTok{""}\NormalTok{,}
                                                         \StringTok{":"}   \OtherTok{=} \StringTok{""}\NormalTok{))),}
         \AttributeTok{party =} \FunctionTok{case\_when}\NormalTok{(speaker }\SpecialCharTok{==} \StringTok{"Le Pen"} \SpecialCharTok{\textasciitilde{}} \StringTok{"Rassemblement national"}\NormalTok{,}
\NormalTok{                           speaker }\SpecialCharTok{==} \StringTok{"Macron"} \SpecialCharTok{\textasciitilde{}} \StringTok{"Renaissance"}\NormalTok{,}
                           \AttributeTok{.default =} \StringTok{"Journaliste"}\NormalTok{)) }\SpecialCharTok{|\textgreater{}} 
  \FunctionTok{select}\NormalTok{(id, year, date, country, speaker, speaker\_turn, party, text)}



\CommentTok{\# Enregistrer la base de données}

\FunctionTok{write\_csv}\NormalTok{(Data\_debate\_2017, }\StringTok{"/Users/oliviasaffioti/Desktop/fas\_1001\_Saffioti/\_tp/TP3/debat\_macron\_lepen.csv"}\NormalTok{)}


\CommentTok{\#Nettoyer la base de données}

\FunctionTok{library}\NormalTok{(quanteda)}
\end{Highlighting}
\end{Shaded}

\begin{verbatim}
Package version: 3.3.1
Unicode version: 14.0
ICU version: 71.1
Parallel computing: 8 of 8 threads used.
See https://quanteda.io for tutorials and examples.
\end{verbatim}

\begin{Shaded}
\begin{Highlighting}[]
\FunctionTok{library}\NormalTok{(clessnverse)}
\end{Highlighting}
\end{Shaded}

\begin{verbatim}
DISCLAIMER: As of July 2023, `clessnverse` is no longer under active development.
To avoid breaking dependencies, the package remains available "as is" with no warranty of any kind.
\end{verbatim}

\begin{Shaded}
\begin{Highlighting}[]
\FunctionTok{library}\NormalTok{(tidyverse)}

\NormalTok{debat\_macron\_lepen }\OtherTok{\textless{}{-}} \FunctionTok{read.csv}\NormalTok{(}\StringTok{"debat\_macron\_lepen.csv"}\NormalTok{)}

\NormalTok{debat\_macron\_lepen}\SpecialCharTok{$}\NormalTok{text }\OtherTok{\textless{}{-}} \FunctionTok{tolower}\NormalTok{(debat\_macron\_lepen}\SpecialCharTok{$}\NormalTok{text)}
\NormalTok{debat\_macron\_lepen}\SpecialCharTok{$}\NormalTok{text }\OtherTok{\textless{}{-}} \FunctionTok{gsub}\NormalTok{(}\StringTok{"[[:punct:]]"}\NormalTok{, }\StringTok{""}\NormalTok{, debat\_macron\_lepen}\SpecialCharTok{$}\NormalTok{text)}
\NormalTok{debat\_macron\_lepen }\OtherTok{\textless{}{-}}\NormalTok{debat\_macron\_lepen }\SpecialCharTok{\%\textgreater{}\%} \FunctionTok{select}\NormalTok{ (id, speaker, text) }\SpecialCharTok{\%\textgreater{}\%} \FunctionTok{filter}\NormalTok{(speaker }\SpecialCharTok{\%in\%} \FunctionTok{c}\NormalTok{(}\StringTok{"Le Pen"}\NormalTok{, }\StringTok{"Macron"}\NormalTok{))}
\end{Highlighting}
\end{Shaded}




\end{document}
