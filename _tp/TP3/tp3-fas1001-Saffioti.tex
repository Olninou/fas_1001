% Options for packages loaded elsewhere
\PassOptionsToPackage{unicode}{hyperref}
\PassOptionsToPackage{hyphens}{url}
\PassOptionsToPackage{dvipsnames,svgnames,x11names}{xcolor}
%
\documentclass[
  letterpaper,
  DIV=11,
  numbers=noendperiod]{scrartcl}

\usepackage{amsmath,amssymb}
\usepackage{iftex}
\ifPDFTeX
  \usepackage[T1]{fontenc}
  \usepackage[utf8]{inputenc}
  \usepackage{textcomp} % provide euro and other symbols
\else % if luatex or xetex
  \usepackage{unicode-math}
  \defaultfontfeatures{Scale=MatchLowercase}
  \defaultfontfeatures[\rmfamily]{Ligatures=TeX,Scale=1}
\fi
\usepackage{lmodern}
\ifPDFTeX\else  
    % xetex/luatex font selection
\fi
% Use upquote if available, for straight quotes in verbatim environments
\IfFileExists{upquote.sty}{\usepackage{upquote}}{}
\IfFileExists{microtype.sty}{% use microtype if available
  \usepackage[]{microtype}
  \UseMicrotypeSet[protrusion]{basicmath} % disable protrusion for tt fonts
}{}
\makeatletter
\@ifundefined{KOMAClassName}{% if non-KOMA class
  \IfFileExists{parskip.sty}{%
    \usepackage{parskip}
  }{% else
    \setlength{\parindent}{0pt}
    \setlength{\parskip}{6pt plus 2pt minus 1pt}}
}{% if KOMA class
  \KOMAoptions{parskip=half}}
\makeatother
\usepackage{xcolor}
\setlength{\emergencystretch}{3em} % prevent overfull lines
\setcounter{secnumdepth}{-\maxdimen} % remove section numbering
% Make \paragraph and \subparagraph free-standing
\ifx\paragraph\undefined\else
  \let\oldparagraph\paragraph
  \renewcommand{\paragraph}[1]{\oldparagraph{#1}\mbox{}}
\fi
\ifx\subparagraph\undefined\else
  \let\oldsubparagraph\subparagraph
  \renewcommand{\subparagraph}[1]{\oldsubparagraph{#1}\mbox{}}
\fi

\usepackage{color}
\usepackage{fancyvrb}
\newcommand{\VerbBar}{|}
\newcommand{\VERB}{\Verb[commandchars=\\\{\}]}
\DefineVerbatimEnvironment{Highlighting}{Verbatim}{commandchars=\\\{\}}
% Add ',fontsize=\small' for more characters per line
\usepackage{framed}
\definecolor{shadecolor}{RGB}{241,243,245}
\newenvironment{Shaded}{\begin{snugshade}}{\end{snugshade}}
\newcommand{\AlertTok}[1]{\textcolor[rgb]{0.68,0.00,0.00}{#1}}
\newcommand{\AnnotationTok}[1]{\textcolor[rgb]{0.37,0.37,0.37}{#1}}
\newcommand{\AttributeTok}[1]{\textcolor[rgb]{0.40,0.45,0.13}{#1}}
\newcommand{\BaseNTok}[1]{\textcolor[rgb]{0.68,0.00,0.00}{#1}}
\newcommand{\BuiltInTok}[1]{\textcolor[rgb]{0.00,0.23,0.31}{#1}}
\newcommand{\CharTok}[1]{\textcolor[rgb]{0.13,0.47,0.30}{#1}}
\newcommand{\CommentTok}[1]{\textcolor[rgb]{0.37,0.37,0.37}{#1}}
\newcommand{\CommentVarTok}[1]{\textcolor[rgb]{0.37,0.37,0.37}{\textit{#1}}}
\newcommand{\ConstantTok}[1]{\textcolor[rgb]{0.56,0.35,0.01}{#1}}
\newcommand{\ControlFlowTok}[1]{\textcolor[rgb]{0.00,0.23,0.31}{#1}}
\newcommand{\DataTypeTok}[1]{\textcolor[rgb]{0.68,0.00,0.00}{#1}}
\newcommand{\DecValTok}[1]{\textcolor[rgb]{0.68,0.00,0.00}{#1}}
\newcommand{\DocumentationTok}[1]{\textcolor[rgb]{0.37,0.37,0.37}{\textit{#1}}}
\newcommand{\ErrorTok}[1]{\textcolor[rgb]{0.68,0.00,0.00}{#1}}
\newcommand{\ExtensionTok}[1]{\textcolor[rgb]{0.00,0.23,0.31}{#1}}
\newcommand{\FloatTok}[1]{\textcolor[rgb]{0.68,0.00,0.00}{#1}}
\newcommand{\FunctionTok}[1]{\textcolor[rgb]{0.28,0.35,0.67}{#1}}
\newcommand{\ImportTok}[1]{\textcolor[rgb]{0.00,0.46,0.62}{#1}}
\newcommand{\InformationTok}[1]{\textcolor[rgb]{0.37,0.37,0.37}{#1}}
\newcommand{\KeywordTok}[1]{\textcolor[rgb]{0.00,0.23,0.31}{#1}}
\newcommand{\NormalTok}[1]{\textcolor[rgb]{0.00,0.23,0.31}{#1}}
\newcommand{\OperatorTok}[1]{\textcolor[rgb]{0.37,0.37,0.37}{#1}}
\newcommand{\OtherTok}[1]{\textcolor[rgb]{0.00,0.23,0.31}{#1}}
\newcommand{\PreprocessorTok}[1]{\textcolor[rgb]{0.68,0.00,0.00}{#1}}
\newcommand{\RegionMarkerTok}[1]{\textcolor[rgb]{0.00,0.23,0.31}{#1}}
\newcommand{\SpecialCharTok}[1]{\textcolor[rgb]{0.37,0.37,0.37}{#1}}
\newcommand{\SpecialStringTok}[1]{\textcolor[rgb]{0.13,0.47,0.30}{#1}}
\newcommand{\StringTok}[1]{\textcolor[rgb]{0.13,0.47,0.30}{#1}}
\newcommand{\VariableTok}[1]{\textcolor[rgb]{0.07,0.07,0.07}{#1}}
\newcommand{\VerbatimStringTok}[1]{\textcolor[rgb]{0.13,0.47,0.30}{#1}}
\newcommand{\WarningTok}[1]{\textcolor[rgb]{0.37,0.37,0.37}{\textit{#1}}}

\providecommand{\tightlist}{%
  \setlength{\itemsep}{0pt}\setlength{\parskip}{0pt}}\usepackage{longtable,booktabs,array}
\usepackage{calc} % for calculating minipage widths
% Correct order of tables after \paragraph or \subparagraph
\usepackage{etoolbox}
\makeatletter
\patchcmd\longtable{\par}{\if@noskipsec\mbox{}\fi\par}{}{}
\makeatother
% Allow footnotes in longtable head/foot
\IfFileExists{footnotehyper.sty}{\usepackage{footnotehyper}}{\usepackage{footnote}}
\makesavenoteenv{longtable}
\usepackage{graphicx}
\makeatletter
\def\maxwidth{\ifdim\Gin@nat@width>\linewidth\linewidth\else\Gin@nat@width\fi}
\def\maxheight{\ifdim\Gin@nat@height>\textheight\textheight\else\Gin@nat@height\fi}
\makeatother
% Scale images if necessary, so that they will not overflow the page
% margins by default, and it is still possible to overwrite the defaults
% using explicit options in \includegraphics[width, height, ...]{}
\setkeys{Gin}{width=\maxwidth,height=\maxheight,keepaspectratio}
% Set default figure placement to htbp
\makeatletter
\def\fps@figure{htbp}
\makeatother

\KOMAoption{captions}{tableheading}
\makeatletter
\makeatother
\makeatletter
\makeatother
\makeatletter
\@ifpackageloaded{caption}{}{\usepackage{caption}}
\AtBeginDocument{%
\ifdefined\contentsname
  \renewcommand*\contentsname{Table of contents}
\else
  \newcommand\contentsname{Table of contents}
\fi
\ifdefined\listfigurename
  \renewcommand*\listfigurename{List of Figures}
\else
  \newcommand\listfigurename{List of Figures}
\fi
\ifdefined\listtablename
  \renewcommand*\listtablename{List of Tables}
\else
  \newcommand\listtablename{List of Tables}
\fi
\ifdefined\figurename
  \renewcommand*\figurename{Figure}
\else
  \newcommand\figurename{Figure}
\fi
\ifdefined\tablename
  \renewcommand*\tablename{Table}
\else
  \newcommand\tablename{Table}
\fi
}
\@ifpackageloaded{float}{}{\usepackage{float}}
\floatstyle{ruled}
\@ifundefined{c@chapter}{\newfloat{codelisting}{h}{lop}}{\newfloat{codelisting}{h}{lop}[chapter]}
\floatname{codelisting}{Listing}
\newcommand*\listoflistings{\listof{codelisting}{List of Listings}}
\makeatother
\makeatletter
\@ifpackageloaded{caption}{}{\usepackage{caption}}
\@ifpackageloaded{subcaption}{}{\usepackage{subcaption}}
\makeatother
\makeatletter
\@ifpackageloaded{tcolorbox}{}{\usepackage[skins,breakable]{tcolorbox}}
\makeatother
\makeatletter
\@ifundefined{shadecolor}{\definecolor{shadecolor}{rgb}{.97, .97, .97}}
\makeatother
\makeatletter
\makeatother
\makeatletter
\makeatother
\ifLuaTeX
  \usepackage{selnolig}  % disable illegal ligatures
\fi
\IfFileExists{bookmark.sty}{\usepackage{bookmark}}{\usepackage{hyperref}}
\IfFileExists{xurl.sty}{\usepackage{xurl}}{} % add URL line breaks if available
\urlstyle{same} % disable monospaced font for URLs
\hypersetup{
  pdftitle={Étude de l'impact du genre sur la saillance des thématiques politiques féminines dans le discours des politicien(ne)s},
  pdfauthor={Olivia Saffioti},
  colorlinks=true,
  linkcolor={blue},
  filecolor={Maroon},
  citecolor={Blue},
  urlcolor={Blue},
  pdfcreator={LaTeX via pandoc}}

\title{Étude de l'impact du genre sur la saillance des thématiques
politiques féminines dans le discours des politicien(ne)s}
\author{Olivia Saffioti}
\date{}

\begin{document}
\maketitle
\ifdefined\Shaded\renewenvironment{Shaded}{\begin{tcolorbox}[frame hidden, boxrule=0pt, interior hidden, borderline west={3pt}{0pt}{shadecolor}, enhanced, sharp corners, breakable]}{\end{tcolorbox}}\fi

\hypertarget{introduction}{%
\subsection{Introduction}\label{introduction}}

Selon la théorie de la construction sociale du genre, les électeurs
évalueraient les politicien(ne)s en fonction des stéréotypes de genre.
Si les politicien(ne)s n'adaptent par leur stratégie marketing aux
stéréotypes, ils risqueraient un blacklash, soit, des évaluations
négatives de la part des électeurs (Rudman et Fairchild 2004; Coyle
2009, cités dans Grebelsky-Lichtman et Katz 2019). Ainsi les femmes et
les hommes politiques, n'useraient pas des mêmes stratégies marketing
afin de convaincre l'électorat. En effet,~les femmes et les hommes
répondraient souvent aux stéréotypes induits par le rôle attribué à
chaque sexe dans la société. Ils conforteraient ainsi la vision des
électeurs (Fox 1997). Cela incluerait l'image, les caractéristiques
personnelles, les thèmes de campagne, et l'utilisation d'enjeux
politiques spécifiques (Fox 1997). À titre d'exemple, les femmes
politiques seraient perçues par l'électorat comme plus efficaces dans
les domaines de l'éducation, de la santé, de l'environnement, des arts,
de la protection des consommateurs, ou encore dans l'aide à apporter aux
pauvres (Alexander and Andersen 1993; Koch 2000; McDermott 1998, cités
dans Fox et Oxley 2003). À l'inverse, les hommes politiques seraient
considérés comme plus compétents pour résoudre des crises militaires ou
de police, des problèmes d'ordre économique, ou encore des enjeux liés
au commerce. Ils seraient également identifiés comme plus légitimes pour
s'occuper du contrôle de la criminalité ou encore de l'agriculture
(Alexander and Andersen 1993; Koch 2000; McDermott 1998, cités dans Fox
et Oxley 2003). En revanche, ces études ont surtout été menée au sein du
contexte étasunien, et nous n'avons trouvé que peu de recherches qui
visent les stratégies des femmes et des hommes politiques lors des
débats électoraux. Il semble ainsi pertinent d'étudier si les femmes
politiques insistent davantage sur les thématiques «~féminines~»
(éducation, etc.) par rapport à leurs homologues masculins, dans un
débat électoral qui n'est pas étasunien . C'est pourquoi, nous avons
décidé d'étudier le débat électoral français opposant Marine Le Pen et
Emmanuel Macron en 2017. Nous avons choisi ce débat car sa
retranscription était facile d'accès. De plus, sélectionner un débat qui
s'était produit après 2019 aurait biaisé nos résultats. En effet, la
pandémie de Covid-19 demeurait un sujet très discuté par les politiciens
durant les campagnes suivant 2019. Ce contexte temporel ne nous aurait
pas permis de vérifier si les femmes parlaient davantage de la santé que
les hommes. À travers ce débat nous vérifierons si madame Lepen a plus
insisté sur les enjeux de l'éducation, de la famille, de l'État
Providence et de la santé que monsieur Macron. Cela nous permettra de
vérifier si les postulats de la théorie de la construction sociale du
genre sont applicables à d'autres contextes. Pour ce faire, nous
convertirons la retranscription pdf en une base de données. Puis, nous
la nettoierons. Ensuite, nous élaborerons un dictionnaire que nous
appliquerons à notre base de données. Afin de mieux visualiser et
comprendre nos résultats, nous élaborerons un graphique.~

\hypertarget{donnuxe9es-et-muxe9thode}{%
\subsection{Données et méthode}\label{donnuxe9es-et-muxe9thode}}

Afin de mener notre recherche, nous avons utilisé la base de données
portant sur le débat présidentiel entre Marine Le Pen et Emmanuel Macron
de 2017. Nous avons créé cette base de données à partir de la
retranscription pdf du débat électoral (que nous avions trouvé sur le
site internet ``vie-publique.fr''). Nous avons nettoyé le texte en
supprimant des caractères indésirables (qui ont remplacé des alinéas,
etc.). En utilisant les fonctions ``data.frame( )'', ``mutate ( )'',
``str\_squish ( )'', ``str\_replace\_all ( )'', ``str\_extract\_all (
)'', ``select( )',''unnest( )'' et ``slice ( )'', nous avons cherché les
noms des intervenants dans le débat et avons créé la variable
speaker\_name dans laquelle placer le nom des intervenants. Ensuite,
nous avons sélectionné le texte et en avons fait une variable intitulée
``text\_data''. Pour cela, nous avons mobilisé les fonctions
``data.frame( )'', ``mutate( )'', ``str\_squish( )'',
``str\_replace\_all( )'', ``str\_split( )'', ``unnest( )'', et ``slice(
)''. Nous avons fusionné les données des noms des intervenants
(speaker\_names) avec celles du texte afin de créer une nouvelle base de
données. Dans cette nouvelle base de données, nous avons également créé
une variable intitulée ``année'' pour y référer l'année du débat, une
variable nommée ``pays'' pour y référer le pays où le débat a lieu, une
variable intitulée ``id'' pour savoir à quel tour des élections le débat
correspond, une variable nommée ``numéro d'intervention'' pour mettre en
lumière l'ordre auquel les intervenants parlent, et une variable
intitulée ``parti'' afin de référé le parti d'appertnance ou la
profession des intervenants du débat. Pour ce faire, nous avons utilisé
les fonctions ``bind\_cols( )'', ``mutate( )'', ``str\_replace\_all(
)'', ``str\_squish( )'', ``case\_when( )'' et ``select( )''. Nous avons
finalement enregistré notre nouvelle base de données dans notre
ordinateur à l'aide de la fonction ``write\_csv( )''.

Ensuite, nous avons importé cette base de données sur Rstudio afin de la
nettoyer et d'opérer une analyse de dictionnaire. Dans un premier temps,
nous avons créé notre dictionnaire afin de pouvoir analyser la fréquence
de certains mots dans le discours des candidats. Personnaliser notre
dictionnaire nous a permis de faire une analyse qui corrélait avec le
contexte (les thématiques évoquées dans le débat) et notre sujet de
recherche. Cela nous a permis de prendre en compte des mots spécifiques
tels que ``protection sociale'' ou ``quotient''. Les mots du
dictionnaire sont reliés aux thématiques ``féminines'' de l'éducation,
de la santé, de la famille, et de l'État providence (lutte contre la
pauvreté, etc.). Afin de créer le dictionnaire, nous avons mobilisé la
fonction ``list( )'', et la fonction ``dictionnary( )''.

Puis, nous avons nettoyé notre base de données : nous avons retiré les
ponctuations du texte et converti le texte en minuscules. Cela nous
permettait par la suite de faciliter notre analyse de dictionnaire (un
mot présent dans notre dictionnaire en minuscules ne serait pas relevé
par notre analyse de dictionnaire dans le texte s'il comprenait des
majuscules par exemple). Pour cela, nous avons mobilisé les fonctions
``tolower( )'' et ``gsub( )''. Nous avons disposé notre texte nettoyé
dans une nouvelle colonne nommée ``text\_clean''. Ensuite, nous avons
sélectionner les variables que nous voulions conserver dans notre base
de données, afin de retirer les variables intules. Nous avons décidé de
conserver les variables ``party'', ``text\_clean'' et ``speaker''. Dans
la variable ``speaker'' nous avons conservé uniquement deux speakers
(deux catégories) : ``Le Pen'' et ``Macron''. Nous avons choisi de les
conserver car nous nous centrons uniquement sur ces deux candidats dans
le cadre de notre étude. Pour cela, nous avons utilisé les fonctions
``select( )'' et ``filter( )''. Ensuite, nous avons procédé à notre
analyse de dictionnaire, et transféré les résultats dans une nouvelle
base de données intitulée ``debat\_macron\_lepen\_resultat''. Pour ce
faire, nous avons mobilisé la fonction ``run\_dictionary( )'', la
fonction ``bind\_cols( )'', et la fonction ``select( )''. ''
run\_dictionary( )'' nous a permis de ``rouler'' notre dictionnaire afin
qu'il relève les mots dans le texte. La fonction bind\_cols a permis de
conserver les colonnes présentes dans la base de données
``debat\_macron\_lepen'', dans notre nouvelle base de données
``debat\_macron\_lepen\_resultat''. La fonction select( ) nous a permis
de sélectionner les colonnes de la base de données ``données
debat\_macron\_lepen'' que nous ne voulions pas garder dans
``debat\_macron\_lepen\_resultat'' (ici, nous n'avons pas conservé la
variable ``party'', et la variable ``doc\_id'').

\hypertarget{ruxe9sultats}{%
\subsection{Résultats}\label{ruxe9sultats}}

\begin{figure}

{\centering \includegraphics{/Users/oliviasaffioti/Desktop/fas_1001_Saffioti/_tp/TP3/Graphique-TP3.png}

}

\caption{Graphique}

\end{figure}

Les résultats que nous avons obtenus indiquent que Marine Le Pen et
Emmanuel Macron n'auraient pas insisté de la même manière sur les enjeux
de la santé, de la famille, de l'État providence et de l'éducation. En
effet, 49 mots liés à l'État providence ont été relevés dans le discours
de Marine Le Pen. Comparativement, 36 mots ont été relevés dans le
discours d'Emmanuel Macron. 13 mots de plus ont été relevés pour Marine
Le Pen par rapport à Emmanuel Macron. Bien que les deux candidats aient
beaucoup insisté sur cette thématique, il semblerait que madame Le Pen
ait plus insisté sur les éléments liés à l'état-providence que monsieur
Macron. De plus, 28 mots liés à l'éducation ont été soulevés dans les
paroles de Le Pen, contre seulement 7 pour Macron. Il y a donc un écart
important de 21 mots entre Le Pen et Macron. Madame Le Pen aurait donc
plus insisté sur la thématique de l'éducation que monsieur Macron. En
revanche, concernant la thématique de la famille, l'écart entre les deux
candidats n'est pas élevé : il est seulement de 3 mots. En effet, 14
mots ont été relevés dans le discours de madame Le Pen contre 17 pour
monsieur Macron. Il semblerait donc que les deux candidats aient insisté
de manière à peu près équivalente sur la thématique de la famille.
Enfin, concernant la thématique de la santé, 15 mots ont été relevés
dans le cas de madame Le Pen et 20 dans le cas de monsieur Macron. Là
encore, l'écart n'est pas important entre Le Pen et Macron (il est
seulement de 5 mots). Il semblerait que monsieur Macron et madame Lepen
aient insisté de manière à peu près équivalente sur la thématique de la
santé.

Dans l'ensemble, plus de mots liés à la famille et à la santé ont été
relevés pour Emmanuel Macron. En revanche, l'écart avec madame Le Pen
n'est pas élevé. Les écarts de mots pour ces deux thématiques ne sont
pas significatifs : ils ne démontrent pas que Macron aurait plus insisté
que Lepen sur ces deux thématiques. En revanche, l'écart entre Macron et
Lepen est plus prononcé concernant la thématique de l'éducation et de
l'état providence. Par conséquent, Marine Le Pen semble, généralement,
avoir davantage insisté sur les thématiques ``féminines'' par rapport à
Emmanuel Macron. Donc, il semblerait que, dans le cas du débat
présidentiel entre Emmanuel Macron et Marine Lepen de 2017, le postulat
selon lequel les femmes politiques insisteraient davantage sur les
thématiques ``féminines'' (santé, éducation, etc.) est validé.

\hypertarget{conclusion}{%
\subsection{Conclusion}\label{conclusion}}

En somme, nous avons étudié le débat présidentiel de 2017 opposant
Emmanuel Macron à Marine Lepen afin de vérifier si madame Lepen
insistait davantage sur les thématiques ``féminines'' que monsieur
Macron. Les résultats issus de notre analyse de dictionnaire ont montré
que madame Lepen insistait davantage sur les thématiques de
l'État\_providence et de l'éducation. De plus, Macron et Lepen
insistaient avec une fréquence de mots quasi-identique sur la famille et
la santé. Nous avons conclu que, généralement, Marine Lepen avait plus
insisté sur les thématiques féminines qu'Emmanuel Macron dans le cadre
de ce débat électoral. En revanche, notre recherche a une faible
validité externe : il aurait fallu étudier plusieurs débats électoraux
de Marine Lepen pour savoir si la candidate a insisté sur ces
thématiques dans le cadre d'autres débats électoraux. De surcroît, il
aurait été pertinent d'étudier des débats électoraux incluant d'autres
femmes issues de partis politiques différents. Cela aurait permis de
contrôler pour le parti d'appartenance, et donc de vérifier si
l'insistance sur les enjeux féminins dépend du parti politique des
politicien(ne)s (et non du genre).

\hypertarget{bibliographie}{%
\subsection{Bibliographie}\label{bibliographie}}

Fox, Richard Logan. 1997. \emph{Gender Dynamics in Congressional
Elections.} Thousand Oaks: Éditions SAGE.
https://books.google.ca/books?hl=en\&lr=\&id=x2w1BfJwRkAC\&oi=fnd\&pg=PR13\&dq=gender+women+elections\&ots=8RGQYfimmw\&sig=Uo31w\_YYAhrRtOEKSyA6ABiEeWw\#v=onepage\&q=gender\%20women\%20elections\&f=false

Fox, Richard L., et Zoe M. Oxley. 2003. ``Gender Stereotyping in State
Executive Elections: Candidate Selection and Success''. \emph{The
Journal of Politics} 65 (3): 833-50.
https://doi.org/10.1111/1468-2508.00214.

Grebelsky-Lichtman, Tsfira, Roy Katz. 2019. '' When a man debates a
woman : Trump vs.Clinton in the first mixed gender presidential debate
``. \emph{Revue Journal of Gender Studies} 28 (6): 699-719.
https://www.tandfonline.com/doi/full/10.1080/09589236.2019.1566890

\hypertarget{annexe}{%
\subsection{Annexe}\label{annexe}}

\begin{Shaded}
\begin{Highlighting}[]
\CommentTok{\# Convertir le texte pdf en base de données csv : }

\DocumentationTok{\#\# Libraries}

\FunctionTok{library}\NormalTok{(pdftools)}
\end{Highlighting}
\end{Shaded}

\begin{verbatim}
Using poppler version 23.04.0
\end{verbatim}

\begin{Shaded}
\begin{Highlighting}[]
\FunctionTok{library}\NormalTok{(tidyverse)}
\end{Highlighting}
\end{Shaded}

\begin{verbatim}
-- Attaching core tidyverse packages ------------------------ tidyverse 2.0.0 --
v dplyr     1.1.4     v readr     2.1.5
v forcats   1.0.0     v stringr   1.5.1
v ggplot2   3.4.4     v tibble    3.2.1
v lubridate 1.9.3     v tidyr     1.3.0
v purrr     1.0.2     
\end{verbatim}

\begin{verbatim}
-- Conflicts ------------------------------------------ tidyverse_conflicts() --
x dplyr::filter() masks stats::filter()
x dplyr::lag()    masks stats::lag()
i Use the conflicted package (<http://conflicted.r-lib.org/>) to force all conflicts to become errors
\end{verbatim}

\begin{Shaded}
\begin{Highlighting}[]
\FunctionTok{library}\NormalTok{(lubridate)}

\DocumentationTok{\#\# Data}

\NormalTok{text\_1 }\OtherTok{\textless{}{-}} \FunctionTok{suppressMessages}\NormalTok{(pdftools}\SpecialCharTok{::}\FunctionTok{pdf\_text}\NormalTok{(}\StringTok{"/Users/oliviasaffioti/Desktop/fas\_1001\_Saffioti/\_tp/TP3/macron\_vs\_lepen.pdf"}\NormalTok{))}

\DocumentationTok{\#\# Settings}

\NormalTok{split\_vector }\OtherTok{\textless{}{-}} \FunctionTok{paste0}\NormalTok{(}\FunctionTok{c}\NormalTok{(}\StringTok{"Mme Le Pen :"}\NormalTok{, }\StringTok{"Mme Saint{-}Cricq :"}\NormalTok{, }\StringTok{"M. Jakubyszyn :"}\NormalTok{, }\StringTok{"M. Macron :"}\NormalTok{, }\StringTok{"https"}\NormalTok{), }\AttributeTok{collapse =} \StringTok{"|"}\NormalTok{)}


\DocumentationTok{\#\# Nettoyage de code}

\DocumentationTok{\#\# Chercher les nom des personnes concernéés dans le débat}

\NormalTok{speaker\_names }\OtherTok{\textless{}{-}} \FunctionTok{data.frame}\NormalTok{(}\AttributeTok{text =} \FunctionTok{paste0}\NormalTok{(text\_1, }\AttributeTok{collapse =} \StringTok{" "}\NormalTok{)) }\SpecialCharTok{|\textgreater{}}
  \FunctionTok{mutate}\NormalTok{(}\AttributeTok{text =} \FunctionTok{str\_squish}\NormalTok{(}\FunctionTok{str\_replace\_all}\NormalTok{(text, }\FunctionTok{c}\NormalTok{(}\StringTok{"[:punct:]C2[:punct:]AB"}  \OtherTok{=} \StringTok{""}\NormalTok{,}
                                                   \StringTok{"[:punct:]C2[:punct:]BB"}  \OtherTok{=} \StringTok{""}\NormalTok{))),}
         \AttributeTok{text =} \FunctionTok{str\_replace\_all}\NormalTok{(text, }\FunctionTok{c}\NormalTok{(}\StringTok{"Emmanuel Macron :"}       \OtherTok{=} \StringTok{"M. Macron :"}\NormalTok{,}
                                        \StringTok{"Marine Le Pen :"}         \OtherTok{=} \StringTok{"Mme Le Pen :"}\NormalTok{,}
                                        \StringTok{"Christophe Jakubyszyn :"} \OtherTok{=} \StringTok{"M. Jakubyszyn :"}\NormalTok{,}
                                        \StringTok{"Nathalie Saint{-}Cricq :"}  \OtherTok{=} \StringTok{"Mme Saint{-}Cricq :"}\NormalTok{)),}
         \AttributeTok{speaker =} \FunctionTok{str\_extract\_all}\NormalTok{(text, split\_vector)) }\SpecialCharTok{|\textgreater{}} 
  \FunctionTok{select}\NormalTok{(}\SpecialCharTok{{-}}\NormalTok{text) }\SpecialCharTok{|\textgreater{}} 
  \FunctionTok{unnest}\NormalTok{(speaker) }\SpecialCharTok{|\textgreater{}} 
  \FunctionTok{slice}\NormalTok{(}\SpecialCharTok{{-}}\DecValTok{805}\NormalTok{)}
  
\DocumentationTok{\#\# Données de prises de parole (text data)}

\NormalTok{text\_data }\OtherTok{\textless{}{-}} \FunctionTok{data.frame}\NormalTok{(}\AttributeTok{text =} \FunctionTok{paste0}\NormalTok{(text\_1, }\AttributeTok{collapse =} \StringTok{" "}\NormalTok{)) }\SpecialCharTok{|\textgreater{}} 
    \FunctionTok{mutate}\NormalTok{(}\AttributeTok{text =} \FunctionTok{str\_squish}\NormalTok{(}\FunctionTok{str\_replace\_all}\NormalTok{(text, }\FunctionTok{c}\NormalTok{(}\StringTok{"[:punct:]C2[:punct:]AB"}  \OtherTok{=} \StringTok{""}\NormalTok{,}
                                                     \StringTok{"[:punct:]C2[:punct:]BB"}  \OtherTok{=} \StringTok{""}\NormalTok{))),}
           \AttributeTok{text =} \FunctionTok{str\_replace\_all}\NormalTok{(text, }\FunctionTok{c}\NormalTok{(}\StringTok{"Emmanuel Macron :"}       \OtherTok{=} \StringTok{"M. Macron :"}\NormalTok{,}
                                          \StringTok{"Marine Le Pen :"}         \OtherTok{=} \StringTok{"Mme Le Pen :"}\NormalTok{,}
                                          \StringTok{"Christophe Jakubyszyn :"} \OtherTok{=} \StringTok{"M. Jakubyszyn :"}\NormalTok{,}
                                          \StringTok{"Nathalie Saint{-}Cricq :"}  \OtherTok{=} \StringTok{"Mme Saint{-}Cricq :"}\NormalTok{)),}
           \AttributeTok{text =} \FunctionTok{str\_split}\NormalTok{(text, split\_vector)) }\SpecialCharTok{|\textgreater{}}
  \FunctionTok{unnest}\NormalTok{(text) }\SpecialCharTok{|\textgreater{}} 
  \FunctionTok{slice}\NormalTok{(}\SpecialCharTok{{-}}\FunctionTok{c}\NormalTok{(}\DecValTok{1}\NormalTok{,}\DecValTok{806}\NormalTok{)) }

\DocumentationTok{\#\# Fusion des deux variables clés et création de la base de données finale}

\NormalTok{Data\_debate\_2017 }\OtherTok{\textless{}{-}} \FunctionTok{bind\_cols}\NormalTok{(speaker\_names, text\_data) }\SpecialCharTok{|\textgreater{}} 
  \DocumentationTok{\#\# Creation de variables et nettoyage \#\#}
  \FunctionTok{mutate}\NormalTok{(}\AttributeTok{year         =} \DecValTok{2017}\NormalTok{, }
         \AttributeTok{date         =} \FunctionTok{ymd}\NormalTok{(}\StringTok{"2017{-}05{-}03"}\NormalTok{), }
         \AttributeTok{country      =} \StringTok{"France"}\NormalTok{, }
         \AttributeTok{id           =} \StringTok{"Débat présidentiel 2017 second{-}tour"}\NormalTok{,}
         \AttributeTok{speaker\_turn =} \DecValTok{1}\SpecialCharTok{:}\DecValTok{804}\NormalTok{,}
         \AttributeTok{speaker =} \FunctionTok{str\_squish}\NormalTok{(}\FunctionTok{str\_replace\_all}\NormalTok{(speaker, }\FunctionTok{c}\NormalTok{(}\StringTok{"Mme"} \OtherTok{=} \StringTok{""}\NormalTok{,}
                                                         \StringTok{"\^{}M."} \OtherTok{=} \StringTok{""}\NormalTok{,}
                                                         \StringTok{":"}   \OtherTok{=} \StringTok{""}\NormalTok{))),}
         \AttributeTok{party =} \FunctionTok{case\_when}\NormalTok{(speaker }\SpecialCharTok{==} \StringTok{"Le Pen"} \SpecialCharTok{\textasciitilde{}} \StringTok{"Rassemblement national"}\NormalTok{,}
\NormalTok{                           speaker }\SpecialCharTok{==} \StringTok{"Macron"} \SpecialCharTok{\textasciitilde{}} \StringTok{"Renaissance"}\NormalTok{,}
                           \AttributeTok{.default =} \StringTok{"Journaliste"}\NormalTok{)) }\SpecialCharTok{|\textgreater{}} 
  \FunctionTok{select}\NormalTok{(id, year, date, country, speaker, speaker\_turn, party, text)}

\DocumentationTok{\#\#Sauvegarde des données}

\FunctionTok{write\_csv}\NormalTok{(Data\_debate\_2017, }\StringTok{"/Users/oliviasaffioti/Desktop/fas\_1001\_Saffioti/\_tp/TP3/debat\_macron\_lepen.csv"}\NormalTok{)}
\end{Highlighting}
\end{Shaded}

\begin{Shaded}
\begin{Highlighting}[]
\CommentTok{\# Faire notre recherche avec une analyse de dictionnaire}

\DocumentationTok{\#\#Importer la base de données : }

\FunctionTok{setwd}\NormalTok{(}\StringTok{"/Users/oliviasaffioti/Desktop/fas\_1001\_Saffioti/\_tp/TP3"}\NormalTok{)}
\FunctionTok{getwd}\NormalTok{()}
\end{Highlighting}
\end{Shaded}

\begin{verbatim}
[1] "/Users/oliviasaffioti/Desktop/fas_1001_Saffioti/_tp/TP3"
\end{verbatim}

\begin{Shaded}
\begin{Highlighting}[]
\NormalTok{debat\_macron\_lepen }\OtherTok{\textless{}{-}} \FunctionTok{read.csv}\NormalTok{(}\StringTok{"debat\_macron\_lepen.csv"}\NormalTok{)}

\DocumentationTok{\#\# Libraries :}

\FunctionTok{library}\NormalTok{(quanteda)}
\end{Highlighting}
\end{Shaded}

\begin{verbatim}
Package version: 3.3.1
Unicode version: 14.0
ICU version: 71.1
\end{verbatim}

\begin{verbatim}
Parallel computing: 8 of 8 threads used.
\end{verbatim}

\begin{verbatim}
See https://quanteda.io for tutorials and examples.
\end{verbatim}

\begin{Shaded}
\begin{Highlighting}[]
\FunctionTok{library}\NormalTok{(clessnverse)}
\end{Highlighting}
\end{Shaded}

\begin{verbatim}
DISCLAIMER: As of July 2023, `clessnverse` is no longer under active development.
To avoid breaking dependencies, the package remains available "as is" with no warranty of any kind.
\end{verbatim}

\begin{Shaded}
\begin{Highlighting}[]
\FunctionTok{library}\NormalTok{(tidyverse)}

\DocumentationTok{\#\# Création du dictionnaire : }

\NormalTok{Dictionnaire\_combiné }\OtherTok{\textless{}{-}} \FunctionTok{list}\NormalTok{ (Santé }\OtherTok{=} \FunctionTok{c}\NormalTok{(}\StringTok{"santé"}\NormalTok{, }\StringTok{"médecin"}\NormalTok{, }\StringTok{"hôpital"}\NormalTok{, }\StringTok{"maladie"}\NormalTok{, }\StringTok{"patient"}\NormalTok{, }\StringTok{"soins"}\NormalTok{, }\StringTok{"médecins"}\NormalTok{, }\StringTok{"assurance maladie"}\NormalTok{, }\StringTok{"patients"}\NormalTok{, }\StringTok{"malades"}\NormalTok{, }\StringTok{"médicaments"}\NormalTok{, }\StringTok{"aide médicale"}\NormalTok{, }\StringTok{"visites médicales"}\NormalTok{, }\StringTok{"visite médicale"}\NormalTok{), É}\AttributeTok{ducation =} \FunctionTok{c}\NormalTok{(}\StringTok{"collège"}\NormalTok{, }\StringTok{"enseignement"}\NormalTok{, }\StringTok{"élèves"}\NormalTok{, }\StringTok{"classe"}\NormalTok{, }\StringTok{"classes"}\NormalTok{, }\StringTok{"diplôme"}\NormalTok{, }\StringTok{"diplômes"}\NormalTok{, }\StringTok{"lycée"}\NormalTok{, }\StringTok{"université"}\NormalTok{, }\StringTok{"universités"}\NormalTok{, }\StringTok{"universitaire"}\NormalTok{, }\StringTok{"universitaires"}\NormalTok{, }\StringTok{"formation"}\NormalTok{, }\StringTok{"formations"}\NormalTok{, }\StringTok{"apprentissage"}\NormalTok{, }\StringTok{"filière"}\NormalTok{, }\StringTok{"filières"}\NormalTok{, }\StringTok{"professionnelle"}\NormalTok{, }\StringTok{"professionnelles"}\NormalTok{, }\StringTok{"école"}\NormalTok{), É}\AttributeTok{tat\_providence =} \FunctionTok{c}\NormalTok{(}\StringTok{"retraite"}\NormalTok{, }\StringTok{"protection sociale"}\NormalTok{, }\StringTok{"protections sociales"}\NormalTok{, }\StringTok{"salariés"}\NormalTok{, }\StringTok{"chômage"}\NormalTok{, }\StringTok{"chômeurs"}\NormalTok{, }\StringTok{"indemnisation"}\NormalTok{, }\StringTok{"indemnisations"}\NormalTok{, }\StringTok{"pauvreté"}\NormalTok{, }\StringTok{"inégalités"}\NormalTok{, }\StringTok{"allocation"}\NormalTok{, }\StringTok{"allocations"}\NormalTok{, }\StringTok{"quotient"}\NormalTok{, }\StringTok{"revenu"}\NormalTok{, }\StringTok{"aide"}\NormalTok{, }\StringTok{"aides"}\NormalTok{), }\AttributeTok{Famille =} \FunctionTok{c}\NormalTok{(}\StringTok{"enfants"}\NormalTok{, }\StringTok{"enfant"}\NormalTok{, }\StringTok{"famille"}\NormalTok{, }\StringTok{"familles"}\NormalTok{, }\StringTok{"famille nombreuse"}\NormalTok{, }\StringTok{"familles nombreuses"}\NormalTok{, }\StringTok{"parents"}\NormalTok{, }\StringTok{"parent"}\NormalTok{, }\StringTok{"familial"}\NormalTok{, }\StringTok{"familiale"}\NormalTok{, }\StringTok{"familiales"}\NormalTok{, }\StringTok{"familiaux"}\NormalTok{)) }\SpecialCharTok{\%\textgreater{}\%} \FunctionTok{dictionary}\NormalTok{()}


\DocumentationTok{\#\# Nettoyer le texte : enlever la ponctuation et les majuscules du texte}

\NormalTok{debat\_macron\_lepen}\SpecialCharTok{$}\NormalTok{text }\OtherTok{\textless{}{-}} \FunctionTok{tolower}\NormalTok{(debat\_macron\_lepen}\SpecialCharTok{$}\NormalTok{text)}
\NormalTok{debat\_macron\_lepen}\SpecialCharTok{$}\NormalTok{text\_clean }\OtherTok{\textless{}{-}} \FunctionTok{gsub}\NormalTok{(}\StringTok{"[[:punct:]]"}\NormalTok{, }\StringTok{""}\NormalTok{, debat\_macron\_lepen}\SpecialCharTok{$}\NormalTok{text)}


\DocumentationTok{\#\# Créer une nouvelle base de données en sélectionnant les variables utiles}

\NormalTok{debat\_macron\_lepen }\OtherTok{\textless{}{-}}\NormalTok{debat\_macron\_lepen }\SpecialCharTok{\%\textgreater{}\%} \FunctionTok{select}\NormalTok{(speaker, text\_clean, party) }\SpecialCharTok{\%\textgreater{}\%} \FunctionTok{filter}\NormalTok{(speaker }\SpecialCharTok{\%in\%} \FunctionTok{c}\NormalTok{(}\StringTok{"Le Pen"}\NormalTok{, }\StringTok{"Macron"}\NormalTok{))}


\DocumentationTok{\#\# Appliquer le dictionnaire : }

\NormalTok{debat\_macron\_lepen\_resultats }\OtherTok{\textless{}{-}} \FunctionTok{run\_dictionary}\NormalTok{(}\AttributeTok{data =}\NormalTok{ debat\_macron\_lepen, }\AttributeTok{text =}\NormalTok{ text\_clean, }\AttributeTok{dictionary =}\NormalTok{ Dictionnaire\_combiné) }\SpecialCharTok{\%\textgreater{}\%} \FunctionTok{bind\_cols}\NormalTok{(debat\_macron\_lepen) }\SpecialCharTok{\%\textgreater{}\%} \FunctionTok{select}\NormalTok{(}\SpecialCharTok{{-}}\FunctionTok{c}\NormalTok{(party, doc\_id))}
\end{Highlighting}
\end{Shaded}

\begin{verbatim}
100% expressions/words found
\end{verbatim}

\begin{verbatim}
0.049 sec elapsed
\end{verbatim}

\begin{Shaded}
\begin{Highlighting}[]
\DocumentationTok{\#\# Faire un graphique en barres : }

\NormalTok{donnees\_graphique }\OtherTok{\textless{}{-}}\NormalTok{ debat\_macron\_lepen\_resultats }\SpecialCharTok{\%\textgreater{}\%} \FunctionTok{group\_by}\NormalTok{(speaker) }\SpecialCharTok{\%\textgreater{}\%} \FunctionTok{summarise}\NormalTok{(santé }\OtherTok{=} \FunctionTok{sum}\NormalTok{(santé), é}\AttributeTok{ducation =} \FunctionTok{sum}\NormalTok{(éducation), }\AttributeTok{famille =} \FunctionTok{sum}\NormalTok{(famille), é}\AttributeTok{tat\_providence =} \FunctionTok{sum}\NormalTok{(état\_providence))}

\NormalTok{donnees\_graphique\_format\_long }\OtherTok{\textless{}{-}}\NormalTok{ tidyr}\SpecialCharTok{::}\FunctionTok{pivot\_longer}\NormalTok{(donnees\_graphique, }\AttributeTok{cols =} \SpecialCharTok{{-}}\NormalTok{speaker, }\AttributeTok{names\_to =} \StringTok{"catégorie"}\NormalTok{, }\AttributeTok{values\_to =} \StringTok{"fréquence"}\NormalTok{)}


\NormalTok{Graphique }\OtherTok{\textless{}{-}} \FunctionTok{ggplot}\NormalTok{(donnees\_graphique\_format\_long, }\FunctionTok{aes}\NormalTok{(}\AttributeTok{x =}\NormalTok{ catégorie, }\AttributeTok{y =}\NormalTok{ fréquence, }\AttributeTok{fill =}\NormalTok{ speaker)) }\SpecialCharTok{+} \FunctionTok{geom\_bar}\NormalTok{(}\AttributeTok{stat =}\StringTok{"identity"}\NormalTok{, }\AttributeTok{position =} \StringTok{"dodge"}\NormalTok{) }\SpecialCharTok{+} \FunctionTok{labs}\NormalTok{(}\AttributeTok{title =} \StringTok{"Fréquence des thématiques discutées par candidat"}\NormalTok{, }\AttributeTok{x =} \StringTok{"Thématique"}\NormalTok{, }\AttributeTok{y =} \StringTok{"Fréquence"}\NormalTok{) }\SpecialCharTok{+} \FunctionTok{scale\_fill\_manual}\NormalTok{ (}\AttributeTok{values =} \FunctionTok{c}\NormalTok{(}\StringTok{"Le Pen"} \OtherTok{=} \StringTok{"lightskyblue"}\NormalTok{, }\StringTok{"Macron"} \OtherTok{=} \StringTok{"lightcoral"}\NormalTok{)) }\SpecialCharTok{+} \FunctionTok{theme\_minimal}\NormalTok{()}


\FunctionTok{print}\NormalTok{(Graphique)}
\end{Highlighting}
\end{Shaded}

\begin{figure}[H]

{\centering \includegraphics{tp3-fas1001-Saffioti_files/figure-pdf/unnamed-chunk-2-1.pdf}

}

\end{figure}



\end{document}
